\section{问题提出的背景}

安全是人与机器人协作中最重要的部分,而底层基于控制的安全算法是整个流程中最后一道屏障。因此安全算法具有重要的意义。如今有多种类型的安全算法,其中基于能量函数的安全控制算法因为其强大的应变能力及效率,在工业界中被广泛使用。

\subsection{背景介绍}

有许多方法可以保证人机交互中的安全,其中一个最常用的方法是通过对机器人运动的底层控制来确保安全。基于控制的算法具有众多优点,例如:1、它不要求复杂的对目标有着精确的预测,甚至某些情况下对于探测的要求也非常低;2、实时运算,快速应对环境变化;3、相比于其他方法,控制算法可以从数学上证明其能确保安全。这些优点使得控制算法成为我们确保人机交互安全的最后一道屏障。在机器人上,即使使用了其他方法进行动作规划,我们也会在底层实现基于控制的算法以确保人类和机器人的安全。因此安全控制算法具有重要意义。

安全控制算法主要分为两类,第一类是防碰撞算法,第二类是碰撞后减伤算法。

防碰撞算法通过约束机器人的动作、规划合理避障路径等方式防止机器人与人类产生物理接触。该类算法具有最好的安全性,但对算法的应变性具有一定要求。

碰撞减伤算法通过实时监测碰撞,在发生碰撞后的第一时间内调整机器人运动趋势以减轻碰撞所造成的伤害,如给予机器人反向运动加速度,触发安全气囊等。

防碰撞算法中的基于能量函数的安全控制算法因其安全、高效、应变能力强等特点备受工业界青睐。而现有的基于能量函数的安全控制算法具有以下共同点:使用能量函数测量安全性,并提供控制输入以使得能量函数维持在较低的值。在不同的方法中,能量函数也被称为势函数,安全指数或屏障函数。这些方法之间的联系和相对优势仍不清楚。

\section{本研究的意义和目的}

我们希望通过统一现有的基于能量函数的安全控制算法,提出一个统一的数学框架,把一种方法表示为一组超参数,这样不同的方法之间进行比较时只需比较超参数即可。定量且直观的展现出了各方法间的区别。

而由于人类运动的随机性,仅仅做数学分析很难估计出算法的真实表现,因此我们还希望实现一个基准评测平台帮助我们对这些方法进行仿真实验,通过大量的仿真得到这些方法的实际表现。

此外,通过理论推导及实验验证,我们希望通过改进现有算法,克服固有缺点,汲取多种方法的长处提出一种具有更好表现的新算法。

该研究可以指出现有安全控制算法的缺陷,并以参数形式具现化个方法之间的不同,为未来该领域的研究给出了指导意见。基准程序平台可用于更为广泛的安全算法比较,包括但不限于基于安全控制的算法。基准平台的出现可以使得该领域的研究者便捷的与现有方法进行比较,通过在多种复杂环境中的仿真验证提出算法的实际效果。