\section{研究计划进度安排及预期目标}

\subsection{进度安排}

进度安排分为四个阶段

3月1日至3月15日:

该阶段专注于学习、总结现有算法,提炼他们的共同点。

3月16日至4月1日:

根据现有算法的共同点提出数学框架,并证明现有方法可以融入到该框架之中。

4月1日至5月15日:

该阶段设计并实现基准评测平台,并将现有算法实现在该平台上,完成实验测评。

5月15日至6月1日:

根据数学分析及实验结果,比较现有算法的优缺点,尝试合并它们的优点,避免缺点,提出一组更好的超参数,即一种更好的算法。

\subsection{预期目标}

该研究具有以下三个目标:

1、提出统一的数学框架,把一种算法表示为该框架下的一组超参数。

2、设计并实现基准评测平台,通过大量的仿真得到不同算法的实际表现。

3、改进现有算法,克服固有缺点,汲取多种方法的长处提出一种具有更好表现的新算法。