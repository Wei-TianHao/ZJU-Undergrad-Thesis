\cleardoublepage

\chapter{文献综述}

\section{背景介绍}

人机交互是一个快速发展的领域,人与机器人的共存、沟通与合作是当今及未来生活与生产中重要的组成部分。人机交互现在常见的形式有:人与机械臂共同装配零件,共用生产线进行生产;自动驾驶车辆与人类司机共享道路;室内机器人助手等。随着工业化程度不断提高以及人工智能的快速成长,越来越多的人机交互是未来发展的趋势。而人机交互之中,安全则是人们考虑的重中之重。由于人机交互可以应用于多个领域,因此本文中使用了“机器人”一词的广义定义。我们把所有机器人类型统称为机器人,例如机械臂,无人驾驶飞机,人形机器人和自动驾驶汽车等。


\subsection{安全的定义}

为了确保人机交互中的安全,我们有必要先了解什么是安全及其各个组成部分。1942年,科幻作家艾萨克·阿西莫夫提出了“机器人三法则”,第一条规定:“机器人不得伤害人类,或者不作为致使人类受到伤害”。受阿西莫夫定义的启发,我们可以定义机器人对人类造成伤害的两种不同方式。
\subsubsection{物理安全}
首先通过直接的身体接触。简单来说,为了使人机交互安全,人与机器人之间不会发生无意识的或可避免的接触。如果对于给定任务需要物理接触(或某些场景下严格防止身体接触既不可能也不实际),施加在人体上的力必须保持低于使人身体不适或受伤的阈值。我们将人机交互中的这种安全形式定义为物理安全。
\subsubsection{心理安全}
然而,单独预防身体伤害并不一定能使得交互过程让人舒适且放松。例如,考虑这样一个场景,机器人使用电锯与人类共同完成任务,程序设定如果人类离得太近则使机器人停机。虽然通过程序设定可以防止直接的身体伤害,但这种交互对人类来说仍有很大的压力。此外,心理上的不适或压力也可能由机器人的外观,凝视,言语,姿势和其他属性引起。
压力可能对健康产生严重的负面影响,这使得压力成为人机交互成为潜在的伤害来源。我们将这种间接的,心理上的伤害的预防定义为心理安全。要注意的是,与身体伤害相比,心理伤害不仅限于近距离交互,它也可能来源于远程交互。

\subsection{安全控制算法}

有许多方法可以保证人机交互中的安全,其中一个最常用的方法是通过对机器人运动的底层控制来确保安全。基于控制的算法具有众多优点,例如:1、它不要求复杂的对目标有着精确的预测,甚至某些情况下对于探测的要求也非常低;2、实时运算,快速应对环境变化;3、相比于其他方法,控制算法可以从数学上证明其能确保安全。这些优点使得控制算法成为我们确保人机交互安全的最后一道屏障。在机器人上,即使使用了其他方法进行动作规划,我们也会在底层实现基于控制的算法以确保人类和机器人的安全。因此安全控制算法具有重要意义。

安全控制算法主要分为两类,第一类是防碰撞算法,第二类是碰撞后减伤算法。

防碰撞算法通过约束机器人的动作、规划合理避障路径等方式防止机器人与人类产生物理接触。该类算法具有最好的安全性,但对算法的应变性具有一定要求。

碰撞减伤算法通过实时监测碰撞,在发生碰撞后的第一时间内调整机器人运动趋势以减轻碰撞所造成的伤害,如给予机器人反向运动加速度,触发安全气囊等。

本文重点关注防碰撞算法。

\section{国内外研究现状}
防碰撞算法有多种实现方式,其中最常见的三类有:1、数值限制;2、速度及距离监测;3、基于能量函数的安全控制。

数值方法通过约束机器人的参数范围,使得机器人不可能对人类产生危害,即使在碰撞发生的情况下依然如此\cite{broquere2008soft}。该方法不需要对外界进行监测,但该类方法可能导致机器人效率低下。

速度及距离监测方法通过设定机器人的安全区以及监测机器人与障碍物的相对位置和速度来防止碰撞,当监测显示障碍物进入安全区时,则停止机器人\cite{lasota2014toward}。该类方法相较于第一类方法有所改进,但面对障碍物仍不够智能。

我们着重介绍基于能量函数的安全控制算法研究进展。

\subsection{研究方向及进展}

基于能量函数的安全控制算法,最早由Khatib与1986年提出\cite{khatib1986real}。该方法定义了一个势能函数来测量安全性,并引导机器人向势能更低(即更安全)的方向运动。势能函数能够根据动态变化的环境因素修改机器人的运动轨迹,因此可以使得机器人做出更复杂、更高效的应对。势能函数的定义形式多种多样,有时被称为安全指数,有时又叫做屏障函数,我们统称该类函数为能量函数。不同方法的能量函数定义不同,同时他们对于给定能量值所做出的应对策略也不同,即对于机器人做出的运动引导不同。比较有代表性的四种方法有势能场算法(PFM)\cite{khatib1986real},滑动模式算法(SMA)\cite{gracia2013reactive},屏障函数算法(BFM)\cite{ames2014control},以及安全集算法(SSA)\cite{liu2014control}。

\subsubsection{PFM}
PFM的灵感源于重力场,它将障碍物视作高峰,机器人的运动视为在一个三维曲面上的运动。可以想象为一个球在凹凸不平的桌子上滚动。由于障碍物处于高出,球在靠近障碍物的时候会受到反向的重力从而远离或绕过障碍物。从而达到安全的目的。

势能场可以不仅局限于二维,我们同样可以把相对速度加入考虑的因素,这样相当于在高维空间的曲面上驱赶一个小球。

\subsubsection{SMA}
SMA的想法是在即将面临危险的时刻给出足够大的运动引导使得机器人仍保持在安全的区域内。具体来说,SMA通过解析算出机器人可以远离障碍物的参数导数,并设定一个足够大的常量,在判定安全的边界情况下,机器人的参数会加上常量乘以参数导数。

\subsubsection{BFM}
BFM的想法是通过解析算出机器人可以远离障碍物的参数导数,并根据当前的危险程度给出一个权重,在下一时刻,机器人的参数会加上权重乘以参数导数。因此BFM在较为不危险的情况下,会给出轻微的运动引导,而在非常危险的情况下,会给出更强烈的运动引导。

\subsubsection{SSA}
SSA是SMA的改进算法,SMA的问题在于运动引导可能过量,导致机器人效率降低。SSA每次给出的运动引导确保能量函数的降低率高于给定的速率,即能够快速的进入安全区,而非使导数变化达到最大值。因此SSA能够在保证安全的同时进一步提高机器人的效率。

\subsection{存在问题}

虽然如今已有众多基于能量函数的安全控制算法,这些方法之间的联系和相对优势仍不清楚。主要体现在一下两个方面

1、不同的两种方法之间的区别是由什么引起的?他们之间又有什么共同点?如何使用最少的修改量使得这两种方法之间相互转化?

2、如何比较不同方法的表现?能否使用理论分析判定一种方法优于其他方法?如果理论分析无法实现那应设计什么样的实验来验证各种方法的表现?

\section{研究展望}

安全算法的基准为题对于算法的开发至关重要,我们既需要理论分析也需要实验统计来比较不同算法之间的优劣。有了基准之后我们可以针对性的改进算法的缺点,并定性定量的分析不同算法的表现差异来源。

% \newpage
\printbibliography[title={参考文献}]

