\cleardoublepage
\chapter{外文翻译}

\section*{摘要}
随着工业化程度提高及人工智能的普及,机器人领域近年来飞速发展。实时导航是机器人领域中重要的研究课题,尤其是在复杂动态环境中的实时导航问题仍存在许多挑战。该领域的发展对于人们的生活及生产具有重要意义。本文介绍了机器人导航领域的研究现状及存在的问题。


\section{总述}

虽然机器人领域近年来飞速发展,但在复杂动态环境中的实时导航问题仍存在许多挑战。例如无人车或无人机在城市中的导航,为了在人口稠密的环境中安全运行,机器人需要有效地应对危险情况并避免任何可能的碰撞。在实践中,这些问题仍难以被解决的原因有:感知能力受限、通信能力不足、计算能力不足、机器人运动能力不足。

因为各行各业对于机器人日益增长的需求,机器人导航问题一直是研究的热点。其他该领域的调查可以在中找到。虽然一蹴而就的解决机器人导航问题可能非常困难,甚至无法实现,研究者们却提出了很多方法解决导航领域中的许多子问题。规划算法可以基于规划的尺度被分为两大类,全局规划(或称为长期规划)和局部规划(或称为短期规划)。

全局规划试图建立一个全面的环境模型,以找到完成导航任务的最佳轨迹。被广泛使用的三种全局规划器有:基于优化的方法(包括模型预测控制),基于搜索的方法(包括A*或D*图搜索)和基于采样的方法(包括各类快速探索随机树算法)。如果已知环境没有任何不确定性,那么全局规划有能力找到全局最优轨迹。然而,在实践中,在给定有限的感知能力的情况下构建环境模型会要求大量计算力。
基于当前感知数据获得的有限环境知识,局部规划试图在短期内仅规划几步。这样的规划器需要较少的计算力,并且是具有有限感知能力的机器人的良好选择。局部规划的一个极端情况是反应控制器,其中规划尺度减少到一步。常见于闭环反应控制器,例如从传感数据到控制的直接映射,这可以极大地减轻计算负担。现有的反应方法包括虚拟力场,势能场和一系列仿生学的方法。然而由于它只在局部调节运动,局部规划更容易陷入局部的最优解。例如机器人可能卡在某个位置而不能到达目标。因此在设计局部规划时,需要考虑对全局目标的影响。然而由于所考虑的系统是非线性的(因为机器人运动受到非完整约束),时间变化(因为系统中的障碍物正在移动和形变),以及随机性(因为环境信息不全)的影响,很难保证局部规划者的全局收敛性。关于环境的信息是有限的。

本文主要介绍了反应式控制器的导航解决方案,该系统考虑了传感器的限制,计算能力以及机器人在移动和静态障碍物混杂的环境中的非完整运动。由于导航的目标可以分为目标到达和避免碰撞,因此反应控制器被设计为两种模式之间的切换控制器。得到的系统表现出滑动模式行为。本文的主要贡献在于,通过在问题几何的某些假设下进行严格的数学分析,证明了所提出的控制器下系统的全局收敛性。

本文的组织以问题为导向,而不是以技术为导向。每一章都是独立的。从第4章开始,每章都涉及一个导航问题。在这些章节中,讨论了问题的模型;提出了控制策略;证明了所提控制策略的避碰和全局收敛的理论保证;并显示了模拟或实验结果。
本文涵盖了从静态环境到动态环境的导航问题,从全局信息到本地信息,从单个机器人到多个机器人。

\section{内容介绍}
第1章到第3章为读者提供了导航问题的背景知识。
第1章介绍了机器人导航在动态不确定环境中面临的挑战以及在被动规划者中使用滑模控制方法的优势。第2章滑模控制的基本原理,回顾了经典滑模控制理论的一些基本概念和事实,它在下面的章节中作为分析开关反应控制定律的数学工具。第3章复杂环境中移动机器人安全导航算法的调查,记录了与无人驾驶车辆导航相关的方法。这些章节的目的是让读者熟悉这一领域的问题和研究。

第4章到第7章讨论静态环境中单个机器人的导航算法。第4章稳定障碍物中轮式移动机器人导航的最短路径算法,考虑了已知环境中全局最短路径规划的问题。使用切线图解决了该问题,该切线图表明这样的最佳路径由自由空间中的直线和障碍物的边界曲线组成。这自然将导航分为两种运动,即自由空间中的直线运动和边界跟随的运动。在此基础上,作者提出了一种在未知环境中机器人导航的反应随机算法,该算法在两种运动之间切换。

第5章用于边境巡逻的轮式机器人的反应导航,提出了具有有限感知能力的边界跟随方法。如前所述,边界跟随可用于在目标到达期间避免碰撞。同时,边界跟随本身很重要,因为它是边境巡逻的主要任务。 %考虑两种实际情况,一种是全向距离传感器,另一种是仅有方向距离传感器。对于这两种情况,
提出了反应滑动模式控制定律以将机器人驱动到距边界预定距离并保持距离。在滑模控制理论的框架下,对所提出的控制律进行了数学上的严格分析。

第6章和第7章分别题为基于边界巡逻算法的未知混杂静态环境中的目标的安全导航和迷宫式环境中非完整机器人的反应导航算法,提出了反应控制策略目标达到不同的静态环境。第6章考虑了杂乱的环境,第7章考虑了类似迷宫的环境。提出的导航策略结合了目标到达模式(直接到目标的运动)和避免碰撞模式(借助于近距离绕过障碍物的边境巡逻算法)。提出了一组规范这两种模式之间切换的规则。然后,作者讨论了策略对障碍物几何形状的要求,以便为避免碰撞和全局收敛提供理论保证。

第8章到第11章讨论了动态环境中单个机器人的导航算法。第8章生物启发的移动障碍物中轮式机器人安全导航算法,提出了一种简单的生物启发策略,用于在动态环境中安全导航Dubins车载机器人,其中障碍物具有平移性动作。
第9章移动和变形障碍中的反应导航:边境巡逻和避免碰撞的问题,提出了一种基于滑动模式的导航和引导类似独轮车的机器人的策略。然后将其应用于巡回移动和变形域的边界并通过动态环境到达目标的问题,所述动态环境混杂有移动和变形的障碍物。

第10章寻找人群中的路径:基于环境综合表示的不知不觉移动障碍物中的机器人导航,提出了一种反应算法,用于在未知的复杂动态环境中移动障碍物的非完整机器人的无碰撞导航。
在所提出的导航算法下,机器人能够通过一群移动或稳定的障碍物寻找短路径,而不是像其他导航算法那样避开人群。
第11章全局聚合反应算法,用于机器人在导航和变形障碍物紧密混乱的情况下导航,介绍并研究了一种纯粹的反应算法,用于在密集混乱的环境中导航平面移动机器人,并且不可预测地移动和变形障碍物。

第12章具有障碍物的未知稳定环境中的多轮机器人的安全协同导航讨论了静态环境中多个机器人的导航算法。它提出了一种基于去中心化、合作、反应和模型预测控制的碰撞避免方案,该方案计划当前感测到的环境部分中的短期路径。在各种场景中的模拟和现实世界测试被用于验证算法。

\section{总结}

本文的目的是介绍动态不确定环境中机器人导航领域的最新进展。局部规划,尤其是使用切换控制法的反应控制器,是本文的重点。作者对碰撞避免行为和提出的控制律的全局收敛行为进行了严格的数学分析,有些是其他文献中也缺失的。
该书可以为学术界和工业界的经验丰富的研究人员提供参考。附录中的材料是独立的,前置要求仅为本科水平的数学基础。读者仅需对编程和机器人技术的有一些基本经验,即可充分理解并实现本文中所提出的算法。读者还可以参考一些便于获取的参考资料,以进一步追求相关的相关主题。不可否认,本文所提出的方法有其自身的局限性,例如对高采样频率的要求和对障碍物几何形状的限制。然而,它们不会妨碍整本文的价值和意义。